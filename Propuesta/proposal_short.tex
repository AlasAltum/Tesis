\documentclass[submission]{eptcs}
\setlength{\parindent}{0pt}
\providecommand{\event}{} % Name of the event you are submitting to

\title{Videojuego educativo para enseñar algoritmos relacionados a grafos}
\author{ Alonso Utreras\\
Profesor Guía: Iván Sipirán\\
GitHub del proyecto: \url{https://github.com/AlasAltum/Tesis}
\institute{Departmento de Ciencias de la Computación\\
Universidad de Chile\\
Santiago, Chile}
\email{alonso.utreras@ug.uchile.cl}
\email{autreras@dcc.uchile.cl}
}

\begin{document}
\maketitle

\section{Introducción}
 
La forma típica de enseñanza, el aprendizaje pasivo, sigue manteniéndose como metodología imperante en las escuelas e instituciones
educativas del país. Sin embargo, esta forma de aprendizaje ha reportado peores resultados. \cite{active_learning_review}. Por otra parte,
existen presiones por parte de las políticas públicas y del mercado laboral que buscan enfocar la enseñanza más en la resolución de problemas
en vez del aprendizaje o memorización de contenidos.

En este trabajo, se busca desarrollar un videojuego donde se muestren grafos, en que el jugador
debe ejecutar las instrucciones de estos algoritmos, apoyándose en elementos visuales. El objetivo de esta investigación de tesis es 
determinar si se ven diferencias en los niveles de motivación percibidos por el estudiantado y en el entendimiento
de los algoritmos, medido a través de tareas de programación. El público objetivo son estudiantes
que estén en su primer año de ciencias de la computación (CS por sus siglas en inglés).

El objetivo de este proyecto es determinar el potencial de los videojuegos educativos, qué tan accesibles son para los usuarios,
qué tanto apoyo pueden presentar en la docencia y probar metodologías de recolección de datos durante el proceso para mejorar el
diseño de videojuegos educativos en el futuro.


\section{Trabajo relacionado} 


En la reseña realizada por Hartikainen et al. \cite{active_learning_review} se dan algunas deficiones de aprendizaje activo:
donde el proceso de enseñanza ocurre porque quien aprende debe ejectuar pasos y participar en alguna actividad a través de la cual se construye el conocimiento. 
El equipo enumera justificaciones para el aprendizaje activo:  mejores resultados, recomendaciones políticas y las nuevas demandas de la vida laboral actual,
como capacidades de comunicación o descubrimiento por cuenta propia. 

Los videojuegos serios (Serious Gaming) son una forma de aprendizaje activo. Un trabajo de Bell y Gibson \cite{evaluation_of_games_for_teaching_cs} indican que los juegos
educativos son más efectivos que las clases, lecturas, videos y tareas. Se indica que el uso de juegos resulta en 9\% mejor retención, 11\% mejor conocimiento declarativo,
14\% mejor conocimiento basado en habilidades y otros indicadores. Sin embargo, recomiendan que los juegos deben ser acompañados
de otras formas de enseñanza, así como hacer actividades post juego donde se le pregunta al estudiantado cómo se relacionan los juegos a la materia.

Bell y Gibson \cite{evaluation_of_games_for_teaching_cs} identificaron y clasificaron videosjuegos de Ciencias de la Computación (CS),
considerando un total de 41 videojuegos. Uno de ellos, Map Coloring, se trata sobre grafos, tomando el tema de coloreo de grafos.
Se realizó una búsqueda del juego a la fecha (2022), pero no se encontró ningún material al respecto.

En el trabajo realizado por Zhao y Shute \cite{video_game_foster_computational_thinking} se enumeran ejemplos de videojuegos pensados para enseñar programación,
tales como Wu's Castle \cite{wuscastle}, CodeCombat \cite{CodeCombat}, CodeSpell \cite{codespells}, MiniColon \cite{minicolon},
tales ejemplos utilizan programación con texto. Sin embargo, también hay numerosos ejemplos que utilizan programación por
bloques, como LightBot \cite{LightBot}, Scratch \cite{maloney2010scratch}, \cite{scratch} y RoboBuilder \cite{RoboBuilder}, los
cuales abstraen el trabajo de aprender una sintaxis relacionada a los lenguajes de programación.
Sin embargo, el impacto de estos videojuegos no ha sido evaluado en muchos casos. En los casos documentados, se cuenta con
muestras pequeñas, además de que se trata principalmente de evaluaciones puramente cualitativas \cite{video_game_foster_computational_thinking},
\cite{effectiveness_gbl}.

Kiili y su equipo \cite{using_videogames_maths} analizaron el uso de videojuegos en enseñanza
y evaluación en matemáticas a través de los títulos "Semideus" y "Wuzzit Trouble". A través
de estos, llegaron a la conclusión de que es posible utilizar videojuegos para enseñar y
evaluar al mismo tiempo. Además, caracterizaron estadísticamente las diferencias producidas por el uso
de videojuegos entre resultados de un pre test y un post test, concluyendo que existe una correlación positiva
entre el uso de estos videojuegos y mejores resultados académicos en pruebas de matemáticas.

Kiili y su equipo, Bell et al., Giani y otros \cite{petri2018method} están de acuerdo en que no hay sistematización en la evaluación de
videojuegos educativos. En efecto, hay juegos serios que ni siquiera se denominan o consideran como tales
\cite{evaluation_of_games_for_teaching_cs}. Por otra parte, no existe una forma estándar de evaluarlos, razón por
la cual Petri, Giani y otros \cite{petri2018method} crearon el modelo MEEGA+ (Model for the Evaluation of Educational
Games and EGameFlow scale).

Entre las faltas mencionadas al momento de crear videojuegos, se mencionan las faltas de 1) Definición de un objetivo
de evaluación; 2) Diseño de investigación; 3) Programa de medición; 4) Instrumentos de recolección de datos y 5) Métodos
de análisis de datos. Un ejemplo de falta de sistematización: una práctica común al analizar estas herramientas son los 
comentarios informales por parte del estudiantado \cite{petri2018method}.


\section{Objetivos e hipótesis}

\subsection{Objetivo generales}

% En la metodología debería especificar lo específico de estos objetivos con las preguntas de investigación
% En la metodología hay que enlazar objetivo con la pregunta de investigación


Diseñar y desarrollar un videojuego que muestre grafos y algoritmos relacionados a estos,
que permita al usuario ir ejecutando acciones en la aplicación según lo pide un algoritmo
desplegado en la misma pantalla. La aplicación debe indicar al usuario cuando está en lo
correcto y cuándo se equivoca.
Se busca establecer diferencias en rendimiento académico y niveles de motivación logrados.

En este caso, se define el rendimiento académico como los resultados obtenidos en una evaluación relacionada a grafos.
La motivación se mide según un formulario formulado por Petri \cite{petri2018meegaplus} que busca medir la percepción de un
videojuego y la motivación que este entrega.


\subsection{Objetivos específicos}

\begin{itemize}

\item Diseñar una aplicación interactiva que muestre grafos y permita seguir
los pasos relacionados a algoritmos que trabajen con grafos.

\item Medir el interés de estudiantes de computación relacionado a grafos
antes y después del uso de esta herramienta a través de un formulario.

\item Medir la comprensión y capacidad de replicación de un algoritmo relacionado a grafos
de un grupo de estudiantes de computación antes y después de probar el videojuego presentado.

\end{itemize}


\subsection{Hipótesis}

Se entienden mejores resultados académicos como mejores notas en evaluaciones estándares
realizadas por los equipos docentes a cargo del ramo.

Se entiende interés como una medida subjetiva de cuánto un usuario desea
aprender sobre un tema en particular.

El grupo de control es aquel que no jugó el videojuego.
El grupo de juego es aquel que probó el juego antes de la evaluación. 


H1: Un videojuego educativo es una herramienta que mejora los resultados en un test de programación 
y niveles de interés en estudiantes de primer año de computación.

Para determinar que los resultados académicos son mejores se tendrá como hipótesis nula:

$$ \bar{x}_{control} = \bar{x}_{juego} $$

Donde $\bar{x}$ es el promedio en la nota de cada grupo.

Para determinar que los niveles de motivación son mayores, se va a suponer que el nivel
de motivación antes y después del videojuego no se ve afectado.

$$ \Delta \bar{Motiv}_{juego} = 0 $$

Con $\bar{Motiv}_{juego}$ como el nivel de motivación base con respecto a la situación inicial, antes de jugar
el videojuego.

\subsection{Preguntas de Investigación}

El grupo de control es aquel que no jugó el videojuego.
El grupo de interés es aquel que probó el juego antes de la evaluación.

P1: ¿Cómo se comparan los resultados en un test de programación entre el grupo de control y el grupo de interés?

P2: ¿Cómo se comparan los niveles de motivación entre el grupo de control y el grupo de interés?


\section{Metodología}


Se desarrolla un videojuego considerando el feedback de un profesor guía y otros estudiantes
que no sean parte de la población objetivo del estudio, ajustándolo según sea necesario. \\
Asegurarse que el videojuego cumple con las características seguridas por autores que proponen
metodologías para la evaluación de videojuegos educativos: Gibson et al. \cite{evaluation_of_games_for_teaching_cs},
Petri, Giano y otros, \cite{petri2018method} y Kiili et al. \cite{using_videogames_maths}. Si no, seguir iterando. \\
Cuando se haya impartido la materia relacionada a grafos y se hayan visto los algoritmos BFS y DFS, preguntarle a
los alumnos si se creen capaces de programar tales algoritmos desde 0 y guardar estos resultados.
Presentar el juego en los cursos cuando se esté impartiendo la materia relacionada a grafos. El uso del
videojuego será voluntario y sugerido, con el requisito de llenar un formulario posterior al uso del juego.
El formulario poseerá un cuestionario siguiendo los lineamientos de MEEGA+ planteados \cite{petri2018meegaplus} \\
Se revisarán los datos llenados por los estudiantes que accedieron al formulario, en conjunto con los
datos recolectados durante el juego mismo, para ver si estos se condicen. Por ejemplo, el juego medirá el tiempo
entre clicks. Se espera que un jugador que esté motivado posea un alto nivel de actividad, y una alta cadencia de clicks. \\
Finalmente, se realizará una prueba de programación en los cursos que vean la materia de grafos, donde los estudiantes
tendrán que responder preguntas relacionadas a los algoritmos vistos en el juego. Se separarán dos grupos, 
uno de control con la gente que decidió no jugar, y otro experimental que sí probó el videojuego.

El programa ofrecerá a los jugadores ejecutar tareas relacionadas con los algoritmos BFS, DFS, Kruskal y Prim relacionados a grafos.
Para completar los desafíos presentados, el usuario deberá realizar las instrucciones dictadas por un computador, invirtiendo los roles
de un programador y un computador. En este caso, el juego limita al jugador por medio de sus reglas, a seguir los pasos del algoritmo 
presentado. Si el jugador ejecuta correctamente ls instrucciones, ganará, pudiendo pasar a otro nivel.

El videojuego guardará datos durante su ejecución y los enviará a alguna base de datos donde se almacenarán para su posterior
análisis. Los datos que se guardarán serán: clicks, movimientos del mouse, acciones del teclado a lo largo del tiempo.
Estos datos se usarán para contrastar con la información final y para determinar el grado de veracidad de las respuestas.


Si algo falla durante el proceso, se contará con los datos en otras facultades en el mundo, así como otros semestres académicos. El juego está
desarrollado en un 70\%, por lo que esto no debería ser problema el desarrollo de este. Se consultó con un comité ético si existen problemas con
la metodología, pero no indicaron que existan complicaciones.


\section{Resultados esperados y discusión}


\subsection{Resultados esperados}

Se espera observar un nivel de interés superior en los estudiantes con respecto a tópicos de algoritmos
relacionados a grafos a través del cuestionario entregado una vez hayan entregado su trabajo.
Se esperan mejores resultados en el grupo que pruebe el videojuego, en un intervalo del
8\% al 14\% de diferencia, lo que debe ser entre 5 y 7 décimas en una escala de evaluación del 1.0 al 7.0
considerando valores decimales intermedios.
Además, se busca generar una metodología de prueba para herramientas interactivas cuyo fin sea la enseñanza,
de tal manera que otro memorista o tesista en el futuro pueda hacer pruebas análogas con otros algoritmos
o estructuras de datos e incluso aplicadas a otras materias.

\subsection{Discusión y debilidades}

El tamaño muestral puede no ser suficiente. No se sabe cuánto será el tamaño de los grupos de control
y experimental, pero se espera que el total de participantes sea superior a 50. \\
Existe un sesgo al presentar el uso del videojuego como una opción voluntaria, por lo que los estudiantes
con más tiempo e interés tenderán a probar el videojuego, pero son quienes a su vez tienen mejores notas. \\
Además, si se presenta solo a estudiantes de una única universidad y un único hemisferio, se presentarán otros sesgos
relacionados a la estacionalidad (primavera). Por lo tanto, para aumentar la robustez de los resultados se 
llevará este experimento a otras facultades donde se impartan ramos de Algoritmos y Estructuras de Datos donde se
aprenda de grafos.

\nocite{*}
\bibliographystyle{eptcs}
\bibliography{generic}


\end{document}
