\documentclass[submission]{eptcs}
\setlength{\parindent}{0pt}
\providecommand{\event}{} % Name of the event you are submitting to

\title{Herramienta interactiva para enseñar sobre Grafos y algoritmos relacionados}
\author{ Alonso Utreras
\institute{Department of Computer Science\\
University of Chile\\
Santiago, Chile}
\email{alonso.utreras@ug.uchile.cl}
\email{autreras@dcc.uchile.cl}
}
\begin{document}
\maketitle

\section{Introduction}
 
Una forma común de enseñar y aprender es acompañar la información con dibujos o gráficos, lo cual
facilita el aprendizaje de quienes tienen una forma de inteligencia más visual. El presente trabajo
busca llevar las instrucciones llevadas a cabo por los algoritmos relacionados con grafos a una
representación visual e interactiva para que los estudiantes de computación puedan aprender acompañándose
con esta herramienta.

\section{Related work}

Parte de los siguientes párrafos están inspirados en el trabajo de José Romero.
Actualmente existen algunas librerías para generar visualizaciones de estructuras de
datos, pero estas no incluyen interactividad. Un ejemplo de librería es Reftree [Stanch, Nick: Reftree.
https://github.com/stanch/reftree, 2021.] que permite crear visualizaciones de estructuras de datos en
el lenguaje Scala.
Otro ejemplo es la librería Lolviz [ 8 ], la cual solo genera imágenes estáticas.
Por otro lado, existe la librería Aed-Utilities [ https://github.com/ivansipiran/aed-utilities ] creada para el curso de Algoritmos
y Estructuras de Datos de la FCFM por el Profesor Iván S.
Además, el estudiante José Romero de la FCFM a cargo del profesor antes mencionado
empezó un trabajo de título relacionado a la visualización de estructuras de datos
a inicios del año 2022. [https://github.com/romero-jose/visualizations]
Un último ejemplo más similar a lo buscado es Manim [Sanderson, Grant: Manim. https://github.com/ManimCommunity/manim, 2021], focalizada en visualizaciones 
matemáticas. Esta librería se utiliza en el contexto de la generación de videos de Youtube,
cuyo fundador Grant Sanderson sube videos a su canal 3Blue1Brown.
Un canal de Youtube que se ha esforzado en hacer representaciones visuales de los algoritmos 
es AlgoExpert 
[https://www.youtube.com/channel/UCoRwjeG00b8vFyATaBOxmOA/featured] o
GeeksForGeeks 
[https://www.youtube.com/c/GeeksforGeeksVideos/], sin embargo,
al tratarse de plataformas de videos, no han entregado interactividad al usuario para
ejercer un rol más activo en su aprendizaje.
Un último ejemplo destacable aplicado a Computación es Scratch, un lenguaje de programación por bloques
hecho especialmente para enseñar programación a infantes y adolescentes a través de visualizaciones y acciones,
ofreciendo gran interactividad [https://scratch.mit.edu/].


\subsection{ Estrategia A}

Texto ...  \cite{hopper2001empirical}. \\


\subsection{ Estrategia B}


\subsection{ Estrategia C}


\section{Problem}

Actualmente no existe una plataforma donde se pueda aplicar
un algoritmo de forma manual e interactiva, lo cual constituye
una buena herramienta de enseñanza. 


\section{Research questions}

P1: ¿Qué tan eficaz es el uso de herramientas de visualización de algoritmos interactivas para
enseñar sobre algoritmos aplicados a grafos?

P2: ¿Cómo se comparan los niveles de interés en el estudiantado al utilizar herramientas
de visualización de algoritmos interactivas comparadas a aquellos grupos que no las utilizan?


\section{Hypothesis}

H1: Una plataforma de visualización interactiva de algoritmos de grafos
es una herramienta que mejora los resultados de los estudiantes que la ocupan.

H2: Una plataforma de visualización interactiva de algoritmos de grafos
genera mayor interés en el estudiantado.

Se entienden mejores resultados como mejores notas en evaluaciones estándares
realizadas por los equipos docentes a cargo del ramo.

Se entiende interés como una medida subjetiva de cuánto un usuario desea
aprender sobre un tema en particular. 

\section{Main Goal}

Diseñar y desarrollar una aplicación interactiva que muestre grafos y 
algoritmos relacionados a estos, que permita al usuario ir ejecutando
acciones en la aplicación según lo pide un algoritmo desplegado en la misma
pantalla. La aplicación debe indicar al usuario cuando está en lo correcto
y cuándo se equivoca.

\section{Specific goals}

\begin{itemize}
\item Diseñar una aplicación interactiva que muestre grafos y permita seguir
los pasos relacionados a algoritmos que trabajen con grafos.

\item Medir el interés de estudiantes de computación relacionado a grafos
antes y después del uso de esta herramienta.

\item Medir la comprensión y capacidad de replicación de un algoritmo de 
estudiantes de computación antesy después del uso de esta herramienta.


\end{itemize}

\section{Methodology}

Diseño de la herramienta interactiva.
Desarrollo de la herramienta interactiva.
Pruebas con casos de estudio.


\section{Expected results}

Se espera aumentar el interés de los estudiantes en el aprendizaje de algoritmos
relacionados a grafos. Además, se busca generar una metodología de prueba para
herramientas interactivas cuyo fin sea la enseñanza, de tal manera que otro
memorista o tesista en el futuro pueda hacer pruebas análogas con otros algoritmos
o estructuras de datos e incluso aplicadas a otras materias.

\nocite{*}
\bibliographystyle{eptcs}
\bibliography{generic}
\end{document}
